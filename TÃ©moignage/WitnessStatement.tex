\documentclass[a4paper]{article}

\usepackage{titling}
\newcommand{\subtitle}[1]{%
  \posttitle{%
    \par\end{center}
    \begin{center}\large#1\end{center}
    \vskip0.5em}%
}

\usepackage[utf8]{inputenc}
\usepackage[T1]{fontenc}
\usepackage{url}
\usepackage{mathtools}
\usepackage{pxfonts}
\usepackage{listings}
\usepackage{xcolor}
\usepackage{graphicx}
\usepackage{array}
\usepackage{caption}
\usepackage[final]{pdfpages}

\title{Witness statement}

\begin{document}

	\maketitle
    \tableofcontents
    \newpage

\section{Sue Exell}

    \let\thefootnote\relax\footnote{\url{http://vol4.royalcommission.vic.gov.au/index03a1.html?pid=111}}

    I own the 4.7 acre property located at <Removed to protect the privacy>, Haven, with my husband Gary. Maps showing the location of our property are attached to this statement and marked SE-1. The bushfire on 7 February 2009 damaged parts of our property and killed or injured some of our pets. Gary and I were at the property at the time and we fought the fires for several hours with the help of some neighbours, friends and CFA volunteers. We were not injured. Other properties near ours and a community recreation reserve were destroyed or badly damaged by the fire.

\subsection{Our property and the surrounding area}

    I have lived in Haven since I was a child and still have a lot of close family in the area. Before building the house on our current property, Gary and I lived at another property which is approximately 500 metres down the road. We built the house on that property in 1978 and it now belongs to Noel and Pauline Keyte.
    Gary and I have lived at our current property since 1995 and we built the house with help from professionals. It is a double-brick house with a tiled roof. Gary is a plumber and we run our plumbing business from home. Our property has a large shed which we use for the plumbing business and we employ eight plumbers and an office manager. A diagram showing the lay out of our property is attached to this statement and marked SE-2\footnote{\url{http://vol4.royalcommission.vic.gov.au/files/SE_2.jpg}}. Aerial photographs of the property are also attached to this statement and marked SE-3\footnote{\url{http://vol4.royalcommission.vic.gov.au/images/witness/exell_sue/SE_3_1.jpg}}\footnote{\url{http://vol4.royalcommission.vic.gov.au/images/witness/exell_sue/SE_3_2.jpg}}. The house is located at the southern end of the property next to a plastic water tank and some large bird cages which housed about 35 pet birds before 7 February 2009. Most of these birds died on the day. We also have an outdoor living area attached to the house which has an in-ground swimming pool which is sheltered by a Colorbond and Laserlite roof. The shed is roughly in the centre of the property and it is attached to five water tanks – four plastic and one steel. There are also two shipping containers beside the shed which are used for storing plumbing materials. Our five plastic water tanks have a 22,500 litre capacity and our steel tank has 4,000 litre capacity. On 7 February 2009 they probably only had about 15,000 litres of water between them.
    Gary and I have two daughters, whose names are Breanne Mills and Kristen Exell. Breanne lives nearby in Grahams Bridge Road with her husband Jeff and their baby daughter Zara. Kristen lives in Queensland. My brother Eric Lane and his wife Anne also live next door to us on the Henty Highway and I have another brother called Kevin Lane who lives nearby in Plozzas Road with his wife Lesley. They have a son Justin Lane who trains horses at their property, and at the time of the fires both Kevin and Justin had gone to collect horses from a property in Golf Course Road.
    Haven is normally an irrigation area which has been badly affected by the drought. I think the drought has greatly increased our risk of bushfire but before 7 February 2009 I don't think that anyone in the community (including me) had realised how much. Before the drought, Haven was always green in the summer and there was always lots of water in the dams and irrigation channels. Gary and I have a small dam at the northern end of our property but it has been several years since it had any water in it during the summer months. Our back fences run along irrigation channels which have been dry for about five years. These channels are basically long ditches which were dug into the ground and they are not lined with concrete. By way of example, a photograph of a similar channel in another part of Haven is attached to this statement and marked SE-4\footnote{\url{http://vol4.royalcommission.vic.gov.au/images/witness/exell_sue/SE_4.jpg}} – the channel can be seen on the left of the photograph. Before the drought started, the channels would be full of water every summer but dam fills were stopped about five years ago and no-one in Haven receives any water allocations anymore. In addition, some of the channels have been neglected and have become overgrown with grass, trees and weeds.

\subsection{No fire plan}

    Gary and I did not have a fire plan before 7 February 2009. I never seriously believed that a bushfire would come to Haven and I had never experienced one before during all the years I had lived there. Although I was aware of the risk with the tea trees behind our property.
    A number of years ago, my sister was living in Greendale (near Bacchus Marsh) and she was nearly burnt out by a bushfire. She told me at the time that it was their pool that had saved the house because they had been able to bucket water out of it. They relied on electric pumps for their water supply. Our water system consists of a permanently installed electric water pumps on our property which we use for pumping water out of our tanks, but we didn't have a portable pump which could be used to get water out of the pool. Again, I never really thought that we would have a bushfire at our place but I nagged Gary into buying a portable petrol-powered water pump. I call this pump the "fire pump". Gary was able to use the pump for business purposes but we never tested it for fire fighting and we also never tested the hoses that came with it. Gary also purchased a petrol-powered generator a couple of years ago so that we would have a back-up electricity supply and running water during normal black-outs and for work. We weren't thinking about fire fighting when we bought the generator.
    Over the years, the CFA in our local area have handed out brochures and organised events about fire fighting but I never really absorbed any of the information or attended the events. I remember hearing about having a "fire plan ready" but Gary and I never really discussed one. I had never been to any CFA training sessions or community meetings before 7 February 2009. After the fires, Jenny McGennisken (who is a friend of mine and a full-time CFA employee) also told me that a fire awareness meeting was held at the Haven Community Hall about two or three years ago but only about three people turned up. I don't recall ever hearing anything about the meeting, but even if I had I may not have gone.
    In hindsight, I have realised that the drought has changed Haven and I was a bit blasé about it. I never really realised that our risk of being affected by a bushfire had increased until after we had the fire. I think that I just took it for granted that because we had never had a major fire before, we would never have one. Black Saturday was a real kick in the pants for the whole Haven community and we have a lot of work to do to make sure that the place is safer in the future.

\subsection{Black Saturday}

    I am usually very busy on Saturday mornings and 7 February 2009 was no different. I am the President of the Haven Tennis Club and I help organise senior and junior tennis competitions every Saturday. I also co-ordinate a community market in Haven once a month and the market was scheduled for that day. It is a typical market and the vendors sell things like organic vegetables, home-made cakes and biscuits and locally grown honey. We typically have between 30-50 stalls each month. The tennis competitions and the market are both held at the Haven Recreational Reserve, right next to one another, and the tennis club sells cooked breakfasts to the market customers.
    During the week leading up to 7 February 2009, the weather was very hot and Jenny McGennisken from the CFA told me that the heat on 7 February 2009 would be extreme and that bushfire conditions were expected to be worse than they were on Ash Wednesday. Jenny is also the secretary of the Central Wimmera Tennis Association, which my club plays under, so she gave my club and the other clubs that information so that we could decide whether to cancel our tennis for the day. The previous Saturday had also been very hot and both the senior and junior tennis competition that day were cancelled. Because of the warning Jenny gave the junior committee during the week, we cancelled the junior tennis competition for 7 February 2009. I was pleased about that because both the junior and senior competitions involve about 140 players on the day. Some of the CWTA committee also wanted to cancel the senior competition but we could not convince all of the committee to do it in advance. Some wanted to stick to competition bylaws which allow for a decision to be made at 12.30pm.
    Another local market at Jung (about 15 kilometres to the north) was held the previous Saturday. During the week leading up to 7 February 2009, I called a few of the stall owners to find out whether the Haven Market should be cancelled because of the weather but everyone I spoke to said that they wanted to go ahead. They said that they had everything ready and that some orders had already been made. I did not cancel the market and I figured that if it was hot, people would come early.
    On 7 February 2009, I arrived at the Haven Recreational Reserve at about 7.00am to open up the buildings and help set up for the market. The tennis club started serving breakfast at about 8.00am. I remember thinking that the weather was warm but not unpleasant at that time and the wind wasn't too bad.
    At about 9.30am, the wind started to pick up and it got much hotter. By that stage the market had started and I was surprised that quite a few people had still turned up. At about 10.00am, the temperature had hit 40 degrees Celsius and the senior tennis competition was cancelled. At about 10.30am an elderly neighbour <This copy has been removed to protect the privacy of an individual.>, who lives alone next to the Recreational Reserve, came over and told me that the wind had blown the roof off her garage. She asked if someone could come and help her so I organised a few blokes to go over to her house to get her car out of the garage and clean up a bit. That was the first sign that the weather was going to be really bad.
    The wind was also blowing over the market stalls and making life very difficult for the stall owners. By about 10.30am I decided to advise the stall owners that if they wanted to get going early that would be fine and that their site fees would be transferred to the next Haven market. The stall owners packed their things and the last one was gone by about midday. The fire later burnt through the part of the Haven Recreational Reserve where the market and the tennis competitions are held so I was very relieved that everyone had cleared out early.
    When everyone had left the market, I went home and checked on my birds. I put a big blind over the cage to give them some shade. I had a really uncomfortable feeling about the day but there were no signs yet of any fire. I went inside and had some lunch with Gary and while we were eating, I suggested to Gary to get the fire pump out of the shed just in case. Gary just looked at me funny and said that he was going to go into the shed to clean up a few things. Gary went to put his boots on and while he was doing that, I looked out the window and saw some hazy smoke to the north-west. Gary said that he thought it was just dust but we went outside and straight away we noticed that we could smell smoke. It was about 12.45pm when we smelt the smoke and as soon as that happened, Gary agreed to go and get the fire pump.
    Even though we could smell the smoke, we weren't sure where the fire was. We had seen smoke off in the distance many times before when there were fires in the Grampians or Little Desert but they were miles away from Haven. I remember that on one occasion, we even had falling ash at the property but the fire didn't come anywhere near us. Having said that, this was the first time that I had ever smelt smoke like this and I was more worried than I had ever been before. While Gary was getting the fire pump ready, I checked the CFA website but I couldn't find anything that related to our district. I never thought about turning the radio on and I did not listen to the radio at any time on 7 February 2009.
    I quickly gave up on the internet and I thought about calling Jenny McGennisken. I went to get the phone but before I reached it, it rang and it was Jenny calling me. Jenny just quickly said her name, that the fire had jumped Remlaw Road (to our north-west), that it was heading towards Haven and Wonwondah and that I should put my fire plan into action. Gary and I didn't have a fire plan so we thought about things to do on the spot. I thought I had dressed properly for fire fighting because I had put on long pants and sleeves, but Gary pointed out to me later that my pants were made of polyester and the material of my shirt was very thin.
    After receiving Jenny's call, I took everything (including doormats) off the veranda and threw some things inside and some in the pool. I ran inside and checked that our windows were closed. While I was cleaning up, Gary set up our generator (we thought that we could lose power if the fire hit our road) and he got some hoses out and connected them up to our garden taps around the house. He also set up the fire pump next to the pool and unravelled the fire hoses which came with it. We had never tested the two hoses before and when Gary unravelled them he realised that they were only about four metres long each, so he quickly had to join them together to make one longer hose. It was just lucky for us that Gary is a plumber and Jenny's warning had given Gary had enough time to connect the hoses.
    Once I had finished taking things off the veranda, I spent about 15 minutes making phone calls, answering calls and rushing around outside helping Gary hose down the house. I called the following people and passed on the message from Jenny that the fire was definitely coming to Haven:
    \begin{itemize}
        \item my daughter Breanne and her husband Jeff – Breanne then took Zara to Jeff's parents' house in Kalkee (north of Horsham) and Jeff himself came to our place to help out;
        \item my brother Eric who lives next door – he said he had also smelt the smoke and had heard a fire truck go past;
        \item my brother Kevin who lives around the corner – his wife Lesley answered the phone and said they knew about the fires and that Kevin and their son Justin had gone to another property to round up their horses;
        \item our neighbours Kevin and Sandra Wallis – Kevin normally helps at the market but instead had gone to Rainbow (approximately 100km north of Horsham) early. When Kevin answered he said that they was at Pimpinio (approximately 17km north of Horsham) and on their way home;
        \item our neighbours Noel and Pauline Keyte who live in our old house 500 metres down the road – I asked Noel if he could go and get Annie Gould and keep her safe at their place; and
        \item our other next door neighbour (on our south side), Neville Pope – Neville didn't answer so I just left a message that the fire was coming through their fence.
    \end{itemize}
    While I was running around hosing the house and making these phone calls, I kept both my mobile phone and our cordless landline phone in my pockets. I also received some phone calls from concerned friends who wanted to know how we were going but eventually I couldn't keep answering the calls because I was too busy.
    Jeff arrived at our house at about 1.00pm and from that point onwards I lost track of time because the fire front arrived a few minutes afterwards. I remember that by the time Jeff arrived, the smoke had become very thick and it was difficult to see and breathe. Until that happened, I had never given any thought to the fact that thick smoke would make it difficult to breathe but when it did happen, I went inside and got some of Gary's hankies to cover our noses. When I went inside all of our smoke alarms were going off. The hankies were not long enough to tie around the backs of our heads so I went back inside and got some wet tea towels instead and they worked better. The wind was also extremely strong and I noticed later that in the day that some of the trees in our local area had snapped in half. That is, they hadn't been pulled out of the ground – they were mature trees and their trunks had just snapped in half.
    Shortly after Jeff arrived, some other people called in to help. Two of our neighbours, Mark Hutchinson and Riley McFarlane, were the next to arrive. Riley is only about 17 and he was only wearing a singlet and shorts. They said they had been driving past to go somewhere else but decided to pull in because they could see the flames coming. Jeff also called his brothers who are CFA volunteers and they were also able to assist. After they left more CFA volunteers arrived in their trucks. I also remember that DSE vehicles arrived but the DSE workers did not help us fight the fire – I don't know what they were doing there but they looked like they were just observing. Noel Keyte also told me later that the DSE sat in front of his house while he was fighting a losing battle to save his shed and they didn't help him either or do anything to make him stop and move somewhere safer.
    The fire front came from the north-west and it struck our property at the back fence adjacent to the shed and the shipping containers. Gary was on the roof of the house at that stage trying to keep the leaf litter in our gutters wet. I remember hearing him shout out that the fire had arrived and to ring the neighbour. The property behind our place (which burnt immediately before the fire front reached us) is an open paddock which normally has horses on it, so the grass was short. The flames that we could see were not very big. The flames jumped the irrigation channels which run along our back fence. They seemed to split the fire front in half and direct it around the shed and around our house. I think the shed and shipping containers provided a buffer for our house. So the fiercest part of the fire front didn't actually directly hit our home but went around it in both directions.
    Neville Pope's property next door had a lot of tea trees on it and when the fire reached them it really took off and the flames became much larger. A photograph of the burnt trees on Neville's property (taken later) is attached to this statement and marked SE-5\footnote{\url{http://vol4.royalcommission.vic.gov.au/images/witness/exell_sue/SE_5.jpg}}. I feel sure that Neville's house would have been destroyed but a water-drop helicopter flew over at the critical moment and dumped water on the house. Other neighbours were not so lucky and three houses on the south side of Neville's property from our house were destroyed.
    Back at our place, although the fire front had gone around our house, there were spot fires everywhere on our property. We worked for three or four hours to try and put them out. We have some gum trees near the bird cages and everyone worked hard to keep them wet and stop them from catching fire. Despite that, it must have become too hot inside the cages for the birds because I discovered later that most of them died. Gary was also running behind the shed with a hose to protect the shed. I was part of a group working on bucketing and hosing spot fires which were burning in our garden beds where we had wooden sleepers, mulch and bark chips. We repeatedly put these fires out and then they would reignite after a few minutes. Eventually we decided that water was not doing the job on its own and we raked the mulch and bark apart so this would stop happening. We had the same problem with a large wood pile and we had to pull that apart too. To do that, we needed an excavator. We own an excavator, which at the time was at the Haven Primary School preparing for re-building. I drove Jeff across the road about 300 metres to get it. A photograph of our burnt out and broken up wood pile is attached to this statement and marked SE-6\footnote{\url{http://vol4.royalcommission.vic.gov.au/images/witness/exell_sue/SE_6.jpg}}. Most of our buckets were plastic. At about 2.30pm, and while we were working on bucketing spot fires, one of the CFA volunteers who was helping us picked up a bucket which had softened in the heat but still looked normal. His hand was burnt with hot melted plastic when he picked up the bucket and he later required hospital treatment. Gary tried to get to Horsham to get some petrol for our generator but realised that he would not be allowed back in through the roadblock so he just came back home again.
    At about 3.30pm, my brother Eric came over and told us that he and his wife Anne had gone out into one of their paddocks with their private tanker hoping to cut off the fire before it reached their house and other buildings but embers had gone over the top of them and got into their haystacks in the shed. They fought that fire but eventually lost the shed and nearly everything in it, and they also blew up their fire pump in the process. Anne had also been taken to hospital with heat exhaustion and dehydration. Gary fitted up our petrol pump for Eric and he went back to his place.
    We continued working on defending our property until about 4.00pm, when we realised that we had put out the bulk of the spot fires. Some of the blokes who were helping us left to see if they could help other people and I continued monitoring the property and putting out small spot fires which reignited. I also checked a few things just outside our property and noticed that a power line had come down so I called Powercor to let them know. By about 5.00pm, I thought that our property was safe. During the afternoon Jeff's dad Alan Mills also arrived with his private tanker. Later, he and Jeff went over to Neville Pope's place and put out some spot fires which were still burning there, and then went to other properties as well to help out.
    At about 7.30pm, Gary and I decided that because our power had gone out, we should get our frozen food out of the freezer and take it around to Breanne's and Jeff's house in Grahams Bridge Road. By that time, Jeff had told us that their house had not been affected by the fire. We drove there via the Henty Highway and then Hunts Road. When we arrived there, we had something to eat and we then waited for Breanne and Zara to get home from Kalkee. We left there at about 9.30pm but it was difficult to get home because police roadblocks had been set up on the Henty Highway at the corner of Mackies Road. To get around the road block, we did a big loop via Wards Road, Three Bridge Road and Plozzas Road before getting back to the Henty Highway. When we finally arrived home, we were relieved to find that the spot fires hadn't re-ignited but shocked to see that I had actually left a window open in the house for the whole day – fortunately at the safer end of the house. Having been away for a few hours, I had become anxious that we might have lost the house after fighting so hard to keep it. We should never have left in the first place.
    Once we were home, Gary and I went to bed but it was difficult to sleep because of the fire trucks, vehicles, Powercor equipment and trees falling. I remember that power was restored at about 3.00am because all the lights came back on.
    I realised later that Gary and I did lots of things wrong while we were fighting the fire. These include:
    \begin{itemize}
        \item using plastic buckets – we should have used metal buckets;
        \item not testing the length of the hoses on our fire pump;
        \item leaving a window open in the house while the fires were burning;
        \item storing our other hoses on plumbers' utes instead of making them easily available around the house;
        \item leaving the house so soon after we had finished fighting the fire;
        \item not checking the radio for emergency information;
        \item wearing inappropriate clothing and not preparing any face masks or something else to help us breathe through the smoke;
        \item not catching the cat and throwing him inside to keep him safe (he escaped the property and I thought he would die but he returned home with burnt paws); and
        \item forgetting to remove an old door mat at the shed, which nearly caused the shed to burn down.
    \end{itemize}
\subsection{After Black Saturday}

    On 8 February 2009, a few of our friends came over to help us clean up. The house was full of black soot and ash.
    During the following weeks, I have spent many hours helping to organise the recovery effort in Haven as well as a series of community events. One of the things I helped organise was a large community information meeting at the Haven Community Hall. About 250 people came to the meeting and my role was to help organise all the background planning, food, assisting with audio equipment and the seating for the meeting. This was only one of the numerous meetings which I helped to organise. My work during the recovery period also included the following:
    \begin{itemize}
        \item meeting with local politicians and reporting on how the community was holding up;
        \item participating in working bees at properties where fencing and sheds had been damaged;
        \item participating in ABC radio reports on how the bush fire had affected Haven;
        \item meeting with the local council to discuss and organise cleanup work on affected properties;
        \item distributing information to the community regarding clean up assistance which was available;
        \item meeting insurance assessors who needed to check the damage done to the Haven Recreational Reserve; and
        \item helping organise a "Thank You" concert to raise money for the CFA volunteers, Horsham Red Cross Unit and State Emergency Services and a Family Day for the community.
    \end{itemize}
    SUE EXELL

\section{Gregory Weir}

    \let\thefootnote\relax\footnote{\url{http://vol4.royalcommission.vic.gov.au/indexc08a.html?pid=164}}

    I, Gregory Weir, of <This copy has been removed to protect the privacy of an individual.>, Churchill, in the state of Victoria, state as follows:
    I live at <This copy has been removed to protect the privacy of an individual.>, Churchill with my wife, Angela, and my four children, Dylan, Hayley, Dustin and Hayden.
    I am currently a full time carer for my wife. Previously I was employed as a truck driver.
    I made a statement to police regarding the events of 7 February 2009. It is included in this statement as attachment GW-1\footnote{\url{http://vol4.royalcommission.vic.gov.au/files/GW-1.pdf}}.

    \subsection{Mum's property}

        My mother, Denise Weir, lives at 1090 Jeeralang North Road, Jeeralang North. Mum's property is about 15 acres. It is a hobby farm. I keep alpacas and a goat up there, and Mum keeps chickens and peacocks.
        I also had many belongings stored on Mum's property in four sheds. I am a hoarder by nature, but at my home in Churchill I did not even have a carport so I stored many things at Mum's place. I had three eight tonne truck loads of belongings delivered to my Mum's place when I moved into my current home. My belongings included the fit out of a cafe that I used to own, \$50,000 - \$60,000 worth of tools, including mechanics tools from my Dad's trucking company, and stationary engines that I collected. I also had a 1942 caravan that I had fully refurnished. I also stored family photos and personal papers in the sheds.
        During summer we made sure that Mum's property was completely clear of any ground fuel. At the beginning of every summer we would cut the paddocks down to a lawn and make sure there was not a stick or leaf on the ground. However, there is not much you can do to reduce the fire risk when you have 15 acres clear but solid bush all around you.
        Mum's house is a 30 year old treated pine log cabin. It has been varnished many times over the years and it is almost shiny. The house survived the Black Saturday fires. One theory I have heard is that it survived because the varnish may have reflected the heat enough that it was not an ignition point. I also think that the shade cloth that covered the front balcony at a 45 degree angle to the gutters helped in that it blew embers up and over the flat roof of the house.

    \subsection{Black Saturday}

        On 7 February 2009 I was at home with my family. We were inside the house as it was so hot. We could hear the helicopters overhead and I made the comment, "well, if Elvis is here now, we're in trouble". My Mum had been to the shops and she came to our place and said she could see smoke up in the Jeeralang Hills. I think this was about 1.30 pm.
        We all stepped outside and looked up at the Jeeralang Hills and I could see smoke coming up from the bottom of Thompson's Road area. When I saw the smoke I grabbed my overalls and fire fighting gear to go up to Mum's property. I did not intend to stay and fight the fire; I just wanted to pack up some things and get the property ready for the fire. I knew we had a few hours before the fire came and I wanted to give the alpacas half a chance to survive and try and get the caravan and the gear out and everything. It was also Mum's plan to leave early before the fire came.
        When I was leaving my eldest son Dylan, who was 14 at the time, told me he was coming with me. There was no arguing with him, he was coming up to his grandma's house.
        When we headed up to Mum's property there was just a little fire at the bottom of the hills and I didn't think there were any major dramas. We were just going up to the property to get everything ready in case the fire reached the property. I went up to the property with Dylan and Mum came up with her partner Jack Dekker. When we got to Mum's property we could see the fire was growing and that it was going to be a big fire.
        We mucked around for about an hour preparing the property. We did things like shut shed doors, and for a while I tried to round up the alpacas to take them off the property. I also hooked the caravan up to the car. Mum and Jack left during the afternoon and went to my house in Churchill.
        I saw Mum's neighbour, Paul, during this time. His house is very close to Mum's house; the two houses are about 50 metres apart. When I saw him he was running a sprinkler system that he had on his roof. The neighbour stayed during the fire with his father. Paul told me later that, before the fire came, he left the choke on the water pump so it died after 30 seconds and he had no power. Then, one of his water lines in the roof burst – causing 5000 litres of water to flood his kitchen floor. When the fire front came through, they ran out of the house, leaving doors open in the shed, and ended up lying on the ground just at the back of the house while the fire went over.
        After about an hour at the house we looked over to a communications tower on the north-west ridge which is about two kilometres away as the crow flies. We could see the smoke coming up over that ridge. I wasn't too worried at that stage as the smoke was still two valleys away. We watched the fire come up to the communications tower. We could see that they were fighting the fire really hard up there to stop it getting to the tower. We could see helicopters and bulldozers. From what I could see they managed to knock the fire off the top of the hill and they saved the tower. But the fire kept going around the back of the hill and then swept into Callignee. I think the fire reached Callignee an hour to an hour and a half before it hit us. However, when I saw them knock the fire off the top of the hill I was quietly confident that we were going to be okay then. I thought that the fire was going towards Yarram.
        I knew the wind change was coming from the radio. During the day I had two or three radios on, one which was tuned to 774 and one which was tuned to the local commercial station Gold 1242. They kept on talking about the westerly that was coming. The westerly turned the fire back around into Callignee but it still didn't affect us. But what I think happened is that the westerly wind skipped the fire across into the bottom of our valley.
        While I was trying to herd the alpacas Dylan said to me, "look at that Dad", and pointed at the valley. I looked and saw there was thick smoke. Before I saw that smoke I was still planning on leaving the property even though there was a tree and two big tree limbs across the driveway. I had two chainsaws so I thought we could drive out as long as it was clear. However, as soon as I saw the smoke I decided it would be too dangerous to drive the 5km bush track out.
        I then prepared to face the fire. I unhooked the caravan and put the cars in the paddock, as well as the tractor (in case we needed it later on), and Jack's Hot Rod. I opened all the gas tanks to drain all the gas and put them in the paddock.
        I then dragged equipment down to the dam that we could use during the fire. This included fire extinguishers, blankets, hessian bags and buckets. I placed three or four piles of clothing and equipment around the property so that if we were forced to retreat to different areas we would have equipment. The stockpiles had four or five litres of water, a couple of pairs of overalls and a couple of t-shirts. In the end all of these piles burnt.
        After that there was nothing much we could do except to wait for the fire. We still had power on at the house so we went into the house and put the air conditioner on and drank litres of water and tried to cool down. I spent a lot of time talking to my son at that stage and telling him what was actually going to happen and how bad it was going to get. I said to him, "Dylan, I might tell you to do things you don't want to do but if I say jump you just have to ask how high". As a 14 year-old boy, he turned into a man that day. He is my hero. He had my back all the way and didn't argue with me. I asked him to do some pretty scary things, like run through a wall of fire at one stage to get back to the dam and he followed me straight through it.
        I rang 000 at different times during the day as did my Mum, my wife and my kids. No calls got through as far as I know. I was calling 000 as all I wanted was one load of water to drop on the dam to give us a better chance of survival.
        During the day I was listening to the radio but from what I could see of the smoke around the valley the updates on the radio were always behind. They would mention a flare up somewhere and I would think, "I saw that 20 minutes ago". It seemed like the radio was always half an hour to an hour behind.
        We were standing on the veranda and we could see the fire starting to come up the bottom of the valley. We started to put all our gear back on (overalls, etc) and then Dylan told me there were fires in the paddock. We finished putting our gear back on and started fighting the spot fires in the paddock. We managed to put out the first few but then I turned around there was maybe 100 or 200 spot fires, just a sea of them. I said to Dylan that we should go to the dam because we couldn't do anymore there. We worked our way around to the dam wall and stood on the wall for a couple of minutes. Dylan took some photos with his phone. Then I told him we had to go in.
        The dam is about 7 or 8 metres by 15 metres and it was about a metre and a half deep. It had lily pads all over it and we bucketed these over our heads, as well as the mud and slime that come with them. They helped protect us from the embers. Our overalls are pockmarked from embers bring driven into us by the wind but we didn't get any burns. Dylan's head was singed a little bit where the slime didn't cover it.
        While we were sheltering in the dam one of Paul's large welding bottles exploded sending sections of the bottle to within ten feet of where Dylan and I were.
        There was one stage when we were in the dam when I thought we were in really big trouble because we couldn't actually breathe properly. It was like sucking on a hairdryer – if you tried to take a breath, it was just too hot and you could only take really shallow breaths.
        It was so hot in the fire that the plastic breather in the middle of my face mask melted and the liquid plastic burnt my lips. I grabbed my sunglasses at one stage and they actually squashed and melted in my hands and the lenses fell out when I pulled them off my face.
        While we were in the dam, when I got a chance I ran across to the fence line to yell across at the neighbours to check whether they were alive. The fence was about 100 to 150 metres away. I thought the neighbours were dead because the last I had seen of them was them running around the house while we were in the dam. The neighbour's house was totally engulfed in horizontal flames and a paddock 30 metres from his house that was 4 feet high in feed had flames that were a hundred foot high because of strong northerly winds that just didn't stop blowing.
        I ran across to the fence three or four times to check whether they were alive and the gusts of wind were physically lifting me off the ground and knocking me to the ground. It would do that to me two or three times. On one of these runs I got caught and had to hide behind a tractor wheel until the fire died down again and then sprint back to the dam. It was the wind that was the killer. I've seen some cyclones up in northern Western Australia and it was a cyclonic wind. Once the fire got behind the wind I think it created its own windstorm. At the height of the storm it seemed to come through in waves.
        Eventually the fire passed through and we were able to get out of the dam. We walked to the bottom of the driveway and I could see about 50 metres of the road and there was ten trees across that section of the road. So I knew we could not get out by road. We just had to sit in the paddock and wait. I moved the car to the top of the hill so it could be seen from the road but we couldn't get the car anywhere near the road without cutting fences and things like that because the driveway was blocked as well.
        During this time my family were told that we were dead. The CFA told them that there was nothing left on top of the hill – everything was gone.
        I had a CB radio in the car that I used to try and get in contact with people. After a couple of hours I managed to get onto a person who was driving on the Princes Highway in Traralgon. He pulled over and while he talked to me, his wife, who was in the vehicle with him, called my wife to let her know that we were alive.
        We sat in the paddock until about 11.00 or 11.30pm, and then we could hear a big machine in the distance. I realised it was an earthmoving machine. It took about half an hour to get up to our section of the road. It was a neighbour from down the road who had gone to the quarry and got a tally handler. He cleared the road for Mr and Mrs <This copy has been removed to protect the privacy of an individual.>'s two daughters who were desperate to get to their parent's house, which was up above Mum's property. He also continued clearing the road up past the <This copy has been removed to protect the privacy of an individual.> house.
        Once the road was cleared we were able to drive out. We still had to drive out over huge logs but it was too dangerous to stop once we started driving because there were still trees crashing down everywhere.
        It was very surreal driving out because I had driven that road a thousand times and it was solid bush but after the fire you could see for kilometres through the burning trees and everything was just sparkling. A lot of the big trees had turned into roman candles because the tops had been snapped off and they were sitting there showering sparks like a roman candle two or three metres at the top of a half broken tree.
        We got home a bit before midnight. We probably should have gone to the hospital but my thinking at the time was that there was probably a lot of people worse off than us at the hospital. We both had very sore eyes and throats and found it difficult to see and breathe. That night I slept sitting up with my head hanging down because it was very painful to have the weight of my eyelids on my eyes. I couldn't see properly for three or four days after that. And for two weeks after the fires, Dylan and I were still coughing up black stuff.

    \subsection{After Black Saturday}

        Two alpacas and the goat survived the fires. The alpacas actually survived by sitting on the base of the dam wall and letting the fire come up to them and using their bodies to stop the fire. Alpacas don't run when they panic, they just lay down. And the more scared they get, they stretch their necks right out along the ground and just lie down flat. So the fire came up to them and then stopped. There is a big green patch where they were laying.
        Most of the chickens died as well as all the peacocks that were in the big breeding pen. I was intending to go up and open the gates to give them a greater chance of surviving but I didn't have a chance to get to them. Two peacocks survived out of 30. One was Gonzo and he survived by sitting on the back hand rail of the back balcony. He lost his tail and feathers and his feet were a bit burnt but he survived. He is Mum's favourite – she's had him for ten or 12 years.
        As I mentioned above, the house survived the fire. Embers had actually blown in through gaps in the windows, into the bedrooms. They landed on the pillows and doonas on the beds but nothing ignited. We found embers that had sunk three inches into the pillow and smouldered but not ignited.
        After the fires I was disappointed by the behaviour of some people. I was driving up to the property each day just to make sure that there weren't people snooping around. A lot of stuff was knocked off from properties in the area and I wanted to make sure the property was protected.
        My Mum didn't go back to live at the property until about a month after the fires because at first she found it too upsetting when she went back. She is living back there full time now but she plans to sell the property in the next couple of years once the area has greened up again. She has become too nervous about having to face another fire. Mum has been through fires before as she lived up in the Dandenong Ranges for a long time.

    \subsection{Compensation}

        I lost approximately \$200,000 - \$300,000 worth of belongings in the fire. I hadn't insured any of my belongings that were stored on Mum's property because it was just an interim measure until I moved to another place and put up a bigger shed. Because I don't live on Mum's property I don't get any compensation. I am not even entitled to a case manager. I also cannot get compensation for my tools because I don't have an ABN or for my animals because I am not a farmer.
        I don't agree with how the system of compensation has worked after the fires. I have seen people actually come out in front financially and I don't think that is right. I think the system has been too even in handing out the same amounts of money to people regardless of how much they have lost. I think the system should be based on recompensing people for what they have actually lost.

    \subsection{Roadside Clearing}

        The week before the Black Saturday fires I drove to Boolarra with Jack to check on the progress of the fires there. We drove from Churchill to Boolarra along Grand Ridge Road and about halfway through, I noticed the uncontrolled roadside vegetation through the plantations. The section was about five kilometres long. There were blackberries on the side of the road that were three or four metres thick and then the blackberries were climbing four or five metres up the first row of trees.
        I believe roadside clearing is very important as when a fire hits areas of vegetation that have not been managed, it flares out. The fire might have slowed and would be trickling uphill but if it hits these areas of 30 foot thick dry blackberry, it will flare up again.

    \subsection{Fire fighting experience}

        I have had previous fire fighting experience from living in the Dandenong Ranges and in Churchill. I have also fought a lot of big grass fires in the Western Districts with my father.
        I have never been a member of the CFA officially but during a fire I will go and help out, like filling up tankers, and filling up concrete agitators with water. I have had a lot to do with the Monbulk CFA brigade.
        This fire was a lot scarier than any other fire I have seen. I went through the Ash Wednesday fires when I was 17 or 18. I was in Emerald and it came through very hard and fast there but it was nothing compared to the Black Saturday fires. There was no rhyme or reason to the fire in Churchill. It just didn't make sense.
        \newline
        \newline
        Dated: October 2009
        \newline
        \newline
        GREGORY WEIR

\section{Andrew Kleinig}

        \let\thefootnote\relax\footnote{\url{http://vol4.royalcommission.vic.gov.au/index5446.html?pid=72}}
        I, Andrew Kleinig, of <This copy has been removed to protect the privacy of an individual.>, Ashburton, in the state of Victoria, state as follows:
        Gavin Wigginton and I own the 236-acre property located at <This copy has been removed to protect the privacy of an individual.>, Callignee, Victoria. Maps showing the location of this property are attached to this statement and marked AK-1. The bushfire on 7 February 2009 destroyed our property and everything on it, including a new house and a vast area of native forest. Gavin and I were at the property when the fire front hit but we survived.

    \subsection{Background}

        Gavin and I are not biologically related but our relationship is like that of a father and son. I am 40 years old and Gavin is 63. We first met about 18 years ago while we were both working at the Australian Red Cross Blood Service. After meeting, we discovered that we had a shared interest in the Australian landscape and in environmental conservation. About 12 years ago, we decided to invest together in a piece of land which had a high level of biodiversity value, and we eventually purchased our property in Callignee in 2007. For a relatively short time before Black Saturday, Gavin lived at the property as his permanent residence and I used it occasionally as a holiday house. I live in Ashburton with my partner and our daughter.
        I hold university qualifications in science, specialising in biochemistry and ecology, including units on fire behaviour and management. After completing my ecology studies in about 2001, I began working for Melbourne Water and I have been employed there ever since. I have had various roles with Melbourne Water over the years and my current title is Program Leader – Western Catchments. I am responsible for managing vegetation and natural landscapes on land which surrounds Melbourne's 8500 kilometres of water ways. This work involves protecting native flora and fauna, revegetation, fencing, erosion control, pollution response, removing litter and debris and dealing with various stakeholders, including multiple government agencies. One of the methods I use in carrying out some of this work is controlled burning, particularly in native grasslands because it promotes healthy and biodiverse landscapes. To assist with this work, Melbourne Water has sophisticated fire fighting equipment and employs a team of experienced fire fighters.
        Through my work with Melbourne Water, I have gained extensive experience in the area of fire management and I have a detailed understanding of bushfires. Gavin does not have the same level of professional experience or knowledge about bushfires, but when he moved to Callignee he took steps to inform himself about bushfire risk by consulting the CFA and developing a written fire plan for our property with the assistance of the CFA. I also regularly discussed bushfires with Gavin during this period and I was heavily involved in developing our written fire plan.
        After reflecting on our shared experiences on Black Saturday (described in detail below), Gavin and I realised that there were significant differences between the approaches we each would have taken if we had been alone on the day. For instance, when I saw the fire front approaching I knew straight away that we would not be able to defend our property against it and that despite our fire plan, we were confronted with "Scenario Z" and needed to focus all of our attention on saving our lives. Gavin has told me since that he would have been slower to reach that conclusion because it was his expectation that he was well prepared to fight any bushfire.

    \subsection{Our experience on 7 February 2009}

        On 6 February 2009, Gavin was in Melbourne to celebrate my birthday and to visit his mother, who was elderly and very ill at the time. We went out for dinner that night and Gavin stayed the night at a house he owns in Kew. I stayed with my family at our home in Ashburton. At that time, we were aware of the hot weather and the extreme conditions forecast across Victoria for the following day. In particular, I recall checking the Bureau of Meteorology's website and reading that there would be a south-westerly change late in the afternoon on 7 February 2009 with winds of about 45 kilometres per hour. When I read that, I recall thinking that if there was a wind of that speed and temperatures in the high 40s, the conditions in Callignee would be at the very upper end of what Gavin and I could deal with. I understand that the speed of the winds we experienced on our property during the afternoon on 7 February 2009 was in fact about 100 kilometres per hour. If I had known that we would experience winds of that nature, there is no way that I would have gone to Callignee and I would have done my best to prevent Gavin from going.
        On 7 February 2009, Gavin told me that he received a telephone call at about 11.00am from our neighbour in Callignee, Louise <This copy has been removed to protect the privacy of an individual.>. He said that Louise told him that a bushfire was moving through Gippsland and might threaten our district. She suggested that Gavin return home. I knew that because Gavin was part of a local Community Fire Guard Group with Louise's family and other neighbours, he would feel committed to going back to Callignee to help out and he told me that when we spoke on the phone. Putting that aside, we were both clear in our understanding of our fire plan and we decided to act on it to defend the house as we had previously rehearsed. We discussed the situation and agreed to go the property with Gavin picking me up from my house in Ashburton at some time between 1.00pm and 1.30pm to begin our drive towards Callignee.
        Near Pakenham, we found that the Princes Highway had been blocked by police at the beginning of the Pakenham Bypass. We were very determined to get to Callignee so we took an alternative route, taking the South Gippsland Highway to Korumburra and then the Strzelecki Highway to Morwell. While we drove along the Strzelecki Highway, we saw flames in the distance to the south-east, roughly in the direction of Churchill. In Morwell, we rejoined the Princes Highway and continued to Traralgon and then south to our property. The whole trip took us about three hours and after seeing the police block near Pakenham, we did not see any police or any other authority blocking or diverting traffic.
        We arrived at our property shortly before 5.00pm. There was heavy smoke across our part of Gippsland and we immediately set about implementing our fire plan, which involves a whole set of things that we had rehearsed. In particular, we put on appropriate clothing, put the car in the shed, moved all furniture to the centre of each room, removed essential items from the house to the shed (including Gavin's computer), hosed down the house, and otherwise prepared the house and the surrounding domestic area in accordance with our written fire plan.
        At about 6.30pm, there was an eerie smoke fog and a roaring noise. Then, during the next hour, we experienced the most amazing fire storm. First, as we stood watching on our deck, it started to grow very dark and a few embers fell. Then, on the far horizon to the west, there was a light glow on the ridge below our house which turned to yellow, then orange, and finally red. At about 6.40pm we retreated into the house as a wall of red rapidly grew before our eyes. By about 6.45pm, all the trees around the house were burning and there was a deafening roar. The fire front had arrived at the edge of the clearing which surrounds our house and it skipped the clearing and hit the house almost instantly. The size of this clearing had been the subject of careful planning and we had made it larger than the size required by the Wildfire Management Overlay, with between 25 and 60 metres of bare ground between the house and the bush. Despite that, the clearing seemed to make no difference to the speed at which the fire hit our house, although it might have delayed it subtly and perhaps enough to help save our lives.
        I believe that the fire front struck the property at about 6.45pm because we later discovered a partially melted clock in our shed which had stopped showing that time. When I saw the fire, I was initially mesmerised by the sight of it. Our house had floor-to-ceiling windows and I had a full view of the flames which at a guess would have been at least 30 metres high, moving horizontally, smacking into the side of the house and wrapping around it. It was as though the house had been picked up and thrown into a sea of fire.
        I noticed at about that time that our water pump had failed and it was then that I realised we were in "Scenario Z" – or in other words, that we should not have been there. We were experiencing the worst possible bushfire conditions and all of sudden we only had mops and buckets to fight the fire. In accordance with our fire plan, we retreated into the laundry on the north-eastern corner of the house, being the opposite side from which the fire front had approached. Within two minutes, some of the windows around the house came crashing in, allowing fire to enter the house.
        We stayed in the laundry for approximately 15 minutes, lying on the floor and trying to get oxygen as smoke started to enter the laundry. During this time, we deliberately spoke to one another continuously because it was difficult to breathe and we were worried about losing consciousness. I recall Gavin saying that he could not breathe and at one stage he put his head in a cupboard. I told him firmly to get out of the cupboard but otherwise, most of our talk was commentary on what was happening. I looked at my watch every minute or two and tested the temperature of the air regularly by raising my hand. I remember saying things like "door's hot now, just hang in there another few minutes". Our teamwork during these critical few minutes was vital in saving our lives.
        Towards the end of our stay in the laundry, I recall noticing that the floor tiles were getting hot and it was apparent that the fire had spread to the sub-floor. At about 7.05pm, I was beginning to seriously struggle for breath and the remaining inch of cleaner air low down next to the floor tiles had filled with a whitish layer of thick acrid smoke. I knew we were getting close to losing consciousness. I told Gavin that we should evacuate from the house and I took his hand and led him out of the house and down the back steps which were already on fire. We then ran across to the nearby shed and went inside. The shed had been hit by the fire front but it had not ignited. There was a small fire burning inside which I quickly put out. We slammed the roller doors closed and got in the car where we lay down. By now, the fire front had passed through but the fire had taken hold of the house and the air was still extremely hot.
        While we were in the car, two LPG tanks which were next to the house vented and blasted fire almost onto the shed. It was extremely hot in the car (at least 50 degrees Celsius) and I was sweating profusely. We continued to talk to one another so that we would stay conscious but I started to become really shaky. I told Gavin that I had to get out of the car because otherwise I would pass out. We both spilled out to get a breath of fresher air and Gavin noticed that another small fire had started inside the shed – it appeared to be some plastic which had self ignited. I dragged the burning materials to a clear space and [because I was feeling so weak could only roll over it with my body in order to put it out. We then got back into the car for another few minutes.
        By around 7.25pm, about 15 minutes after we entered the shed, we staggered out of the side door of the shed to find that the house had burnt to the ground but that the fire had subsided and that the radiant heat had reduced significantly. Fires were burning in the trees all around us but it seemed safe enough to be outside. We could not believe that we were alive.
        Approximately 20 minutes later, we heard voices. It was Louise <This copy has been removed to protect the privacy of an individual.> with her husband Tony and their two teenage sons, <This copy has been removed to protect the privacy of an individual.>. They had survived their own inferno about two kilometres up the road. They told us that they had initially sheltered inside their brick house but that the fire had entered the house through a skylight, which they had overlooked when developing their fire plan. They told us that they then retreated into their swimming pool but realised that it would not save them so they quickly moved from there into their dam, where they stayed until it was safe to leave. After debating whether we should leave our property, we walked about three kilometres down the hill and heading north along Red Hill Road. As we walked, we saw burning trees on either side of the road and lying across it. We also saw several neighbours' houses burning and the air was thick with smoke. It was still difficult to breathe – we occasionally choked on the smoke and Gavin needed to stop at one stage for a rest. At the corner of Traralgon-Balook Road, we met a CFA fire tanker and the crew drove us further north to the Traralgon South village.
        We were taken to the Traralgon South CFA fire station where people greeted us, amazed that we had survived. There were a number of government agencies represented there and we reported our names about four times. We were then taken to a relief centre in Traralgon and were eventually sent to a hotel, because the roads in and out of the town were blocked. We were essentially trapped in Traralgon for about four days, and a very kind lady lent us her car for that time.

    \subsection{Treatment by authorities and service providers during the recovery process}

        The support Gavin and I have received from people in the community has been awesome. On the other hand, I have almost never been eligible to receive any assistance from government agencies and service providers, even at the relief centre straight after the fire. On a personal level, people from the various agencies looked like they wanted to help me, but because our Callignee property was not my permanent address and because Gavin and I are not related by blood, it seemed as though I did not fit into any of their categories that would make me eligible for more assistance. There was no appropriate box for anyone to tick and nobody seemed to know what to do with me.
        My experiences in seeking relief were frustrating and somewhat demeaning. From the moment Gavin and I first sought assistance at the relief centre in Traralgon during the evening on 7 February 2009, I was in no man's land. I could not go home to Ashburton because the roads out of Traralgon were blocked and I was only provided with minimal assistance while I was stuck there. In particular, I was not eligible for any financial assistance and had to live off Gavin whilst other arrangements were made. Apart from inviting me to help myself to food, drinks, op-shop clothing and underwear at the relief centres (which I was very grateful for) no-one was prepared to help me. I was not motivated by the prospect of receiving charity, but I had been badly affected by the bushfire and I needed help. I felt like I was being treated like a tourist.
        Months after the fires, I received a \$1000 grant from Centrelink to compensate me for being away from my primary place of residence for over 24 hours, but this was the only assistance I have received from any government agency or service provider.
    \newline
    \newline
    ANDREW KLEINIG

\section{Anja Toikka}
        \footnote{\url{http://vol4.royalcommission.vic.gov.au/index3311.html?pid=175}}
        I, Anja Toikka, of <This copy has been removed to protect the privacy of an individual.>, Callignee, in the state of Victoria, state as follows:
        I am a retired nurse. I am originally from Finland. I came to Australia in 1971 with my then husband. I lived in townships until I moved to Callignee in 1997.
        I bought the land at <This copy has been removed to protect the privacy of an individual.>, Callignee in 1997. I live there with my partner Garry Pratt. I am 70 years old and Garry is 72. I was a 'city girl' before I moved to Callignee, and moving to the country was quite a new and exciting experience.
        I wanted to garden and to create a heaven for myself – and that is what it was for many years. Slowly over many years, Garry and I built it into an absolutely magnificent productive and suitable place for us. I named it "Herran Kukkaro" which means "God's purse" in Finnish. I called it that because it was beautiful and tranquil and the place I wished to be most of all.
        On 7 February 2009, we successfully defended the shed that we lived in on the property, but we lost five other sheds as well as our cars and all of the vegetation on the property. We also lost all out farming equipment, garden tools and implements and a huge amount of building material.

    \subsection{The property}

        My property is just over five acres. The property is hilly, with about 100 metres height difference between the top and bottom. All the adjoining properties are five acres too. I am surrounded by private properties, not farms or forest. A lot of our neighbours have bushland on their properties though. The property that adjoins me to the south had a considerable amount of bushland. Generally however, the surrounding area is cleared farming land with occasional bush, although there are now no commercial farms left. A map of the property is attached to this statement and marked AT-1.
        When I first bought the land, there was a four bedroom house on the property. That house burnt down in 1998, a year after I bought it, after an electrical fire.
        After that, Garry and I built a shed on the foundations of the old house (a concrete slab), which we now live in. We always planned to build a mud brick house on the slab but we did not get to do that before the fires. Garry built the shed and divided it in half and we live in one half. That half is insulated and the windows are double glazed. We have a solar hot water service and a slow combustion stove – it is very comfortable.
        Before the fires we were building a new house in a different spot on the property, higher up in the orchard, in the western corner of the property. It was close to lock up stage at the time of the fires. It did not burn during the fires and we have almost finished it.
        There was a big machinery shed near our living shed which housed our farm equipment and Garry's ute. There was also a double garage to the north where I kept my car, a Toyota Corolla Seca, and the building material we had gathered over the years to build our new house. There were also two garden sheds to the south-west of the living shed where I kept gardening equipment, indoor plants and propagation areas. There was a firewood shed to the south east of the living shed. We also had a wood fired sauna on the property. That got burnt in the fires, so now we have an electric sauna. Photos of the property before the fires are attached to this statement and marked AT-2.
        \footnote{\url{http://vol4.royalcommission.vic.gov.au/images/witness/toikka_anja/AT-2_1.jpg}}
        \footnote{\url{http://vol4.royalcommission.vic.gov.au/images/witness/toikka_anja/AT-2_2.jpg}}
        \footnote{\url{http://vol4.royalcommission.vic.gov.au/images/witness/toikka_anja/AT-2_3.jpg}}
        \footnote{\url{http://vol4.royalcommission.vic.gov.au/images/witness/toikka_anja/AT-2_4.jpg}}
        Along <This copy has been removed to protect the privacy of an individual.> Road the roadside is bordered with old gum trees. They were right on our boundary line but they were on council land. Our orchard is bordered by that road, and large branches used to fall from the gums into the orchard but we were not allowed to trim the trees.
        When I first bought the property there was very little vegetation – it was very bare. When we first moved in, we fenced the whole property, put in a gate, named the property and started gardening.
        The bottom or northern part of the property is near a creek which is a small tributary to Stony Creek and it is reasonably steep so we planted around 2000 blue gum trees there to get firewood in the future. It also meant we wouldn't have to mow or slash that area. Then we divided the rest of the property into little paddocks, for animal husbandry. The top or south eastern part of the property forms a wedge and we kept this area for an orchard. In the orchard we planted apple, nectarine, cherry, apricot and fig trees. There were also olive, citrus, almond, walnut, locust, crap apple, bay leaf, Ginkgo Biloba and Medlar trees. We had over 60 trees and the oldest of them was 25 years old.
        We had 10 chickens and a beautiful white rooster called Kalle. We also have a Shetland pony called Satu (which means fairytale in Finnish) and a cat called Pusser. There were countless wild animals on the property too. Our pony and five of the chickens survived the fires, but the rooster did not. The pony was badly injured in the fires. We have never been able to find Pusser and we don't know if she is dead or alive.
        On the south western side of the property was my large vegetable garden, which was my pride and joy. I love roses so I also had a big rose garden, with 150 plants. When we planted our vegetable garden, we used heritage seeds as much as possible and left them to self sow. I completed a permaculture course in New South Wales many years ago and I devoted myself to that kind of gardening. Some of the vegetables we had in winter included broccoli, cauliflower, cabbage, kale, leek, bok choy, broadbeans and herbs. In summer we had tomatoes, beans, cucumbers, zucchini, pumpkin, corn, capsicum, peas and spring onion. Included in attachment AT-2 are photos of the garden before the fires.
        Before the fires, we bought very little fruit or vegetables during the year. We were almost self-sufficient and got most of our fruits and vegetables from our garden. We also ground our own grains to make our own bread. We also made soap, pickled olives, dried herbs and made jams, chutneys and relishes.

    \subsection{Fire Plan and preparation}

        Because I come from a cold country, I had never witnessed a bushfire first hand, so when I moved to Callignee I thought I had better learn about it. Garry and I went to CFA community meetings for years, and the meetings were very good. We went to about six between 1997 and 2009.
        At these meetings there was an education officer, and also captains and personnel from the brigades who we could ask questions of. The meetings were held at Callignee Hall, next to the fire station. There were never more than five or six people there, but I thought that was because most Australians already knew what bushfires are and they don't need to learn what I needed.
        The meetings helped me to understand how bushfires behave and what would be the best way for us to prepare ourselves. They also taught me how we should behave during a bushfire. One of the lecturers, an Education Officer from the CFA, came and inspected our place. He was quite pleased with our place and the steps we had taken to reduce fuel and clean around our property. He warned us about wood bark mulch so I never had it directly against the shed walls.
        I always believed that if we stayed indoors while the fire front was passing, and had done all the appropriate preparations, we would be safe, providing the dwelling did not catch fire. I was proven wrong. I believe that we were fully prepared when the fire came. We were always going to stay, and I don't think anyone could have changed our minds.
        We tried to do the right thing in regard to reducing fuel load and organising our water supply. We did as much preparation as we could every fire season. We cleaned around the house, we had a firefighting tank, a fire pump, hoses and nozzles. We also had a 1000 litre water tank permanently fixed on a trailer.
        We tested the fire pump several times every season and we had all the appropriate clothing, according to what we had learned at the CFA meetings. We had overalls, woollen beanies, leather boots, leather gloves, cotton or woollen socks, and I sewed together some cotton masks for us to use.
        Water was always a big issue for us in preparation. It is always a balance between how much do I water my garden and how much do I save? We always recycle our water and save as much as we can. We had a 3000 litre swimming pool which we kept filled all year round. It was an above ground inflatable pool. We also had three tanks. One was a big concrete tank that holds about 10,000 gallons, which was our holding tank, and then we have two header tanks which hold about 5000 gallons each. The 10,000 gallon tank was situated at the end of our living shed. We pumped water from that up to the 5000 gallon tanks, which were situated at the highest point on the property. All the pipes, which are plastic, are buried deep underground. The 5000 gallon tanks deliver water to the living shed by gravity feed back. We gravity feed them so that in case of fire, when the electricity goes out, we still have access to that water. The two header tanks burned during the fires and we also had a lot of holes in our pipes after the fires. Even though they were buried deep, some were exposed because we were digging ditches. We lost one whole tank of water because of the holes in the pipes.
        I also knew that I had to prepare myself mentally. I imagined what I would do and how I would behave. I certainly did not prepare for exactly what came but I was preparing for what I thought an "ordinary" bushfire would be like. We had learnt that the initial fire front should go over our house, that it would last for about four or five minutes, and then we would go back outside and start fighting. That is what we had learnt from the meetings we went to.

    \subsection{Lead up to 7 February 2009}

        In February 2009, my son Kari was staying with us for about a month. He is 48 years old and normally lives in Germany. He is a scientist and works for Bosch. He had lived in Australia before but he had never seen a bushfire.
        I was on high alert leading up to 7 February 2009 because the Delburn fires had been burning for about ten days. We are high up on a hill and I could see the smoke column day and night. I took some photos of the smoke column on 29 January 2009 and they are attached to this statement and marked AT-3. It was unsettling and nerve racking. I was also aware of the forecast and I knew the bushfire danger was growing during the week. I also knew there was a wind change forecast for Saturday afternoon. I found that particular forecast very scary. I knew of the forecast from listening to ABC radio and watching the ABC news on television.
        \footnote{\url{http://vol4.royalcommission.vic.gov.au/images/witness/toikka_anja/AT-3_1.jpg}}
        \footnote{\url{http://vol4.royalcommission.vic.gov.au/images/witness/toikka_anja/AT-3_2.jpg}}
        During the week, it got hotter and hotter and it was heart wrenching to watch the garden dying. We were forced to save water and we couldn't water the garden as much as we would have liked.

    \subsection{Black Saturday}

        On the night of Friday 6 February 2009, I did not sleep very much because it was so hot and I was also scared of what was to come. I woke up the next day and thought "Dear God, let us get past this day and I think we will be safe".
        We had breakfast and tried to sort out what we could do that day to entertain my son. We decided that there was no way that we could leave the property and we made it fully clear to my son that we were going to stay if there was a fire. I still blame myself very much for not telling him to get his belongings and go. <This copy has been removed to protect the privacy of an individual.>.
        We were glued to ABC radio during the morning. Their fire reports were very frequent. Kari was also checking the ABC website for fire reports on his laptop. But we only had a dial up connection at that time so the connection was slow. We still cannot get a broadband connection and we have no internet at all right now.
        I did some gardening in the morning. I took a hose out to the front yard so it would be near me. I fed the chooks and let them out of their coop. We all had a cup of coffee around 11.00am and then Garry and Kari decided to go to Mount Tassie to look out for any fires. Mount Tassie is about 10 kilometres from us and it is the highest point on the Strzelecki Mountain Range and so it is good as a lookout point. Mount Tassie has television and radio towers and they got damaged fairly early so we lost our radio. ABC kept broadcasting on a different frequency but we could not get it.
        I did not want the boys to leave because I was scared and unsettled. I had to calm myself when they left and I was trying to think rationally about what I would do if the fire came while I was alone. I went through the same routine in my head that we had practiced every year. All the time in my mind I thought that if it gets really dangerous, if it really starts coming, surely somebody will come and tell us or phone us – or we will get some kind of signal from somewhere.
        Garry told me that when he and Kari got to Mount Tassie they could see that fires were heading south from Churchill towards Yarram. Then there was a wind change while they were up there and they saw the fire starting to come back towards them. After that, Garry said he saw spot fires starting in the gullies between them and the fire. Garry told Kari that they had to go home and get ready because the fire was going to hit us.
        The boys got back home at about 1.00pm and from then I started to understand that nobody would come – it really was up to us then. When the boys got back they told me that Mount Tassie was under ember attack and that the fire was going to hit us.
        After that we put our plan into action. We hitched the trailer with the water tank to the tractor. Garry parked some vehicles, including a front end loader that we had on loan, on the good green grass in the front so it would be easier for us to look after them. We placed mops and buckets in strategic places and rolled out hoses inside our shed. Garry and I dressed ourselves in all our firefighting gear. We also had overalls for Kari. Garry took the pony to a paddock where she would be further from the trees and we thought she would be safest.
        I asked whether anyone wanted lunch and I put together some food, but I don't think any of us ate anything. Kari <This copy has been removed to protect the privacy of an individual.> said to me "Mum we have to go" but I knew that we couldn't go after seeing the fire.
        I think it was just after 4.00pm that the first ember landed on our property but before that happened, the air started to change and it went darker and even hotter. The wind felt like it was moving in all possible directions and when the embers started coming they didn't come just from one direction, they came from everywhere. When the embers first started to land I plunged myself into the pool. It made me feel safer to do that.
        After that there was very little time to think. We each got a mop and water. We mainly scooped the water from the pool and we put out embers everywhere. A couple of times I had to scoop my beanie into the pool again because embers got stuck in it.
        The embers came in all sizes, and sometimes they were lumps of wood flying through the air. There was also a lot of rubbish in the air; there were branches, leaves, and smaller loose items. I noticed that the embers flew into fence posts and lit fires in them straight away because the wood was so dry.
        The main fire front approached us at about 6.00pm. As it was approaching I took photographs. Some of these are attached to this statement and marked AT-4.
        \footnote{\url{http://vol4.royalcommission.vic.gov.au/images/witness/toikka_anja/AT-4_01.jpg}}
        \footnote{\url{http://vol4.royalcommission.vic.gov.au/images/witness/toikka_anja/AT-4_02.jpg}}
        \footnote{\url{http://vol4.royalcommission.vic.gov.au/images/witness/toikka_anja/AT-4_03.jpg}}
        \footnote{...}
        \footnote{\url{http://vol4.royalcommission.vic.gov.au/images/witness/toikka_anja/AT-4_19.jpg}}
        \footnote{\url{http://vol4.royalcommission.vic.gov.au/images/witness/toikka_anja/AT-4_20.jpg}}
        The time date on these photographs is out by an hour because it hadn't reset the clock for daylight saving so the actual time was one hour later than shown.
        The wind was coming from the south west when the embers started but the air whirled around and it was hard to say exactly where it came from. By 6.00pm, the embers died down a bit and it went very very dark – more or less pitch black. And then as the main fire front neared it started getting lighter again. As the fire got closer the colours started changing and it went from dusky yellow to a reddish colour. When the fire front actually arrived it was almost like a sunrise. The whole western sky, as far as one could see, was aflame.
        When the front started coming towards us the pony started screaming. Garry ran to her from the shed and he opened the gate to let her out. The pony ran away to the road in panic.
        Garry then ran from that gate to the entrance to the "other" half of the living shed which is about ten metres and in that time the fire was on top of him. Kari and I were already in there by this time. I was holding a hose. Garry ran in and slammed the shed roller door down. When it was fully down, flames poured in the gap at the top of the roller door. I sprayed the hose onto the flames as well as Garry and I subdued it.
        Garry then got a hose too and we both sprayed all around us. Then I went into our living area where we have a glass door and I could see that fire was coming in the gaps in the walls and ceiling and above the windows. I ran around with a mop in my hand putting out the sparks that were coming in. I lost my breath completely and I had to lie down. While I was lying down I was still using the mop to put out sparks. I thought briefly that we might die but I quickly pushed that aside. I never really lost my hope that we would survive.
        At this time, Garry and Kari were still in the other half of the shed and they were lying on the ground to get more air. The main fire front seemed to last about 20 minutes.
        As I was lying on the ground, I looked out of the double-glazed bay window that was on the front wall facing north. Part of the roof above the window was also glass and in the summer we would put a big shade sail over that part of glass to protect it. I knew that if that window exploded we would be in lots of trouble. But I could see that the shade sail was catching lots of embers in the air and taking the worst of the buffeting. It didn't catch on fire, it only smouldered and melted slowly. I believe that the shade cloth saved the window from exploding and therefore saved our home.
        Garry went outside a couple of times to check on a caravan that we had on loan. Kari was sleeping in this caravan while he was staying. The tyres were on fire so Garry put them out with his mop and bucket and he also put out a burning fence post and wooden garden edging next to it that were on fire. After that it was too hot to stay outside, and he couldn't breathe, so he came back in. Garry went in and out three times putting things out with his mop and bucket and then coming back in when he could hold his breath no longer.
        I think the fire kept coming back on us over and over again as the gum trees along the road kept on exploding and pushing the fire back onto us. I think this because it took a very long time for the front to pass despite the very strong wind pushing it. After about 20 minutes I thought that the worst was over.
        Eventually I could see out the window that the fire front had passed so we could open the roller door. We all then roamed around the property putting out spot fires until 5.00am the next morning. Everything was on fire, including the plants in the garden which was heart breaking. We put out fence posts and tree stumps, plants and wood chip mulch. The inflatable pool was on fire too. I put water on that to put it out but it melted slowly to nothing. By that time we had used most of the 3000 litres of water with our buckets. The buckets and mops were the most useful implements of all in our fight against the fire.
        The big shed that had all the farm equipment and Garry's car was well alight. There was nothing we could do to save it. The garage that had my car in it was well alight too. The fire was so hot that parts of the cars melted and the aluminium tray on Garry's ute ran in rivulets on the ground. All the sheds burnt down including the garden sheds and the wood shed.
        The noise of the fire still haunts me. From the moment the embers came, it got louder and louder. When the fire front was coming it was like there was a jet plane or 10 goods trains coming towards you. The noise was horrific and it became ear splitting. When we first went out to fight the spot fires after the front had gone through you could hear the houses burning and they were so noisy – it was like they screamed. This noise is what I found hard to get over and initially I had repeated nightmares about it.
        At around 5.00am we each had a shower and cleaned up and had a glass of rum. Garry realised then that he had suffered burns through his clothes – that is how hot the fire was. Attached to this statement and marked AT-5 is a photo taken at about that time showing how red Garry's skin had become.

    \subsection{After the fires}

        On Sunday 8 February 2009 we went for a walk along the road which was full of fallen trees, some of them still burning. Four out of five of our neighbours lost their houses. There were dead animals everywhere. Some photos of the destruction that we took that day are attached to this statement and marked AT-6.
        \footnote{\url{http://vol4.royalcommission.vic.gov.au/images/witness/toikka_anja/AT-6_1.jpg}}
        \footnote{\url{http://vol4.royalcommission.vic.gov.au/images/witness/toikka_anja/AT-6_2.jpg}}
        Our rooster died – he didn't burn, he asphyxiated. We found magpies in the yard that were dead but they hadn't burnt or anything either. It seemed like they had dropped out of the sky because their wings were still at full stretch. We found three chooks all together alive and two elsewhere, but two of them had bad burns. One, which was unblemished, died the following day.
        On the following Tuesday night, 10 February 2009, some young people from Traralgon South, who had a horse float found our pony. They had come to our area looking for injured and dead horses. They had to put down quite a few horses but those they deemed to be saveable they tried to return to their owners. Someone told them we had a Shetland pony and they thought she could be ours. When they found her, she was immovable and had to be coaxed into a horse float. She had damage to her lungs and burns on her muzzle, teats and hooves. She also has a spinal injury, probably from something falling on her during the fire. I had to give a bronchial solution into her mouth and throat and give her antibiotics for two weeks. She was very sore, could hardly walk, coughed incessantly and she was clearly in shock. She is a lot better now but she was very sick for a long time. We had assistance from veterinarians for many months. We know she will never be 100% but most of the time she is comfortable and she has also become very loveable since the fires. She is now very gentle and loving.
        One of the hardest things to understand and accept that life will never be the same again. At times the grief for the lost life and resentment and pain for the sudden change forced upon us has been unbearable. There have been many days when I have sat inside and thought "where do I start?" and once I chose something to start with, I found ten other things to do before I could get to that. The whole vastness of the tasks overwhelmed me completely and at the end of each day nothing got done.
        We are pensioners so our income is not very big. We did not have insurance because when we applied some years ago we were refused cover because we live in a shed. Our cars were insured but I was only paid $2000 for mine and Garry got $5000 for his ute. We did get some assistance from VBRRA when they offered assistance for destroyed outbuildings. After the fires, we found out that RACV had changed its rules regarding insuring sheds in November 2008 and that we could have got insurance at that time but we did not know about the change in the rules so we did not apply.
        Garry had also been having health problems before the fire which got worse afterwards. From October 2008, Garry had been in the queue to have a hip replacement which was scheduled for June 2009. He eventually had the operation at the end of July 2009. After the fires, Garry deteriorated quite a lot because of the work that had to be done and also because of the mental strain. This meant that from March 2009, he was on morphine patches constantly because he was in so much pain. Garry had previously had a heart bypass as well, and he has always had blood pressure problems. This was another reason I got so little done because Garry was so sick.
        After the operation at the end of July, Garry was in hospital for a month. He was in there for so long because his rehabilitation was a bit slow but also because our living circumstances are so primitive. When he got out of hospital at the beginning of September, it was the coldest time of the year and we had no wood for our heating. All our firewood had been burnt on Black Saturday as well as our saws and everything we used to cut wood. Garry hadn't been able to do the physical work of cutting more anyway. We were very dependent on neighbours and the community relief centre to give us wood.
        When Garry got out of hospital his hip was still causing him a lot of pain and he couldn't use his hands as he developed bilateral carpal tunnel. At the end of October, he had to have another operation to correct his carpal tunnel and the first fortnight after that was difficult as he had no use of his hands.
        Since Christmas 2009, Garry's health has improved in leaps and bounds and he is pretty fit now physically.
        We have both had problems with healing mentally. Garry started having nightmares again recently which he had straight after the fires as well. I was on antidepressants for a while after the fire and I couldn't sleep. It was because of the fires and also because I was worrying about Garry's health.
        In December, the Rotary Club gave us a holiday. A lady from Tasmania, just outside Hobart, went overseas and needed a house sitter and Rotary paid our airfare to Hobart and St Vincent de Paul paid for a hire car for us. The Tasmanian Rotarians invited us to a dinner meeting and I was asked to speak about our experience of the fires. We knew some of the club members because they had come to our place in April 2009 and built a stable for Satu and donated hay. It was a magical time for us and when we got back we both felt recharged and ready to start again. That was also the first time we had left the property since the fires.
        We have got a lot of help from the local community. A guy came in one day and asked what he could do to help and Garry asked him to help him put in a shed next to the veggie patch. He came back with three of his workmen and built us a shed. He had been flying over the area and said that he thought we had "copped it", and that's why he came in and offered to help us.
        We have both been given cars through relief centres. A car dealer in the Mornington Peninsula heard about me and donated a 20-year-old Ford Laser he had in his car yard. It is a bit battered but he got his mates together and put in new tyres, a new windscreen and a new radiator and they brought it to the Traralgon South Relief Centre. It was totally free. The car dealer even paid for transfer fees. It has over 500,000 kilometres on the clock and I thought that it would stop working straight away but it is still going well.
        Garry also got a ute through the Gippsland Emergency Relief Fund. There was a mine owner in Queensland who donated a 2004 Ford Falcon ute. It is very expensive to drive and very expensive to insure so we generally drive my little car except when we need to tow something. We are very very grateful for the cars we have been given, and for all the assistance we have been given. We are also very grateful for the kindness we have been shown, which is equally important.
        I have noticed that the community has come together differently after the fires. We are all there to help each other. We get together and chat and have meetings every now and then. That sort of thing didn't happen very much before the fires. We have a regular drop in centre now and they are halfway through building a new Callignee community centre which will have facilities for children, the CWA and the cricketers. The fire brigade will be in that centre too.
        However, Callignee is now a very different place in many ways. The vegetation has disappeared and the community is changing. It used to be a farming community but the farms have disappeared and the land divided into smaller acreages. Even then, before the fires, a lot of the houses were old farming houses and it still looked like the countryside. Now a lot of those houses have burnt and people are rebuilding huge multi-storey mansions. Unfortunately, I think that it will be like living in the suburbs in no time. The area is losing its character. It is not the place that we wanted to move to anymore.
        We have continued to build our house and it will soon be ready. The builder has done his bit and it is up to us now to finish it, and we are almost there. It is a little house, just for us. It has insulation in the walls and ceilings, it has double glazed windows, as well as solar power and solar hot water. We will try to keep the principles of bring self sufficient and to be responsible for our own lives and wellbeing.
        We have restarted the veggie garden but it is not the same. It is now in the windiest spot on the property because the wind shelter trees that we had planted to protect it have burnt down. The chook shed is also next to the vegetable garden and they have no shelter now. I have planted new shelter trees but that is a long term project.
        Recently we began hearing bird song again. The balance of nature is still very disturbed and we have a real collection of pests that we didn't use to have before because the control animals aren't there anymore. We used to have a koala that would move around all the time and it died. We also had many different types of frogs and only recently I have seen a few again. There were also an echidna, possums, wallabies and foxes, and none of them have come back.
        I have also noticed that quite a number of our neighbours are breaking up. Garry and I made the choice to come together and work together after the fires. We were offered counselling after the fires and we took up that opportunity. We have a counsellor who comes to our home. Initially he came weekly, and then fortnightly, and now he has been coming every three weeks or monthly. We also did a "mental first aid" course that was run by Lifeline over six weeks and attended Relationship Australia seminars. Neither of us sleep the way we slept before the fires. However, I am now off my anti-depressants and I am managing quite well without them. I don't think we would have done as well without the counselling.
        The fires have changed our lives so completely. In many ways it has been a learning curve. It has been sad but there have been highlights and I have learnt acceptance and how to overcome adversities. I have also learnt that life can continue despite absolute catastrophes. More and more lately I have felt so much stronger and I have a feeling of empowerment that whatever life throws my way I will survive. I have found a strength within me that I might not have ever discovered without this life changing experience.
    \newline
    \newline
    DATED: April 2010
    \newline
    \newline
    ANJA TOIKKA

\section{Rainier Verlaan}

    \footnote{\url{http://vol4.royalcommission.vic.gov.au/index6f49.html?pid=34}}
    I, Rainier Verlaan, of <This copy has been removed to protect the privacy of an individual.> , Callignee, in the state of Victoria say as follows:
    I live at <This copy has been removed to protect the privacy of an individual.>, Callignee with my wife Ann and my daughter, Chelsea, aged 18.
    We have lived in our house since 1997. The house is brick veneer with a colour bond metal roof. Photos of our house before and after the fire are included in this statement as attachment RV-1.
    \footnote{\url{http://vol4.royalcommission.vic.gov.au/images/witness/rainier_verlaan/RV-1_1.jpg}}
    \footnote{\url{http://vol4.royalcommission.vic.gov.au/images/witness/rainier_verlaan/RV-1_2.jpg}}
    \footnote{\url{http://vol4.royalcommission.vic.gov.au/images/witness/rainier_verlaan/RV-1_3.jpg}}
    \footnote{\url{http://vol4.royalcommission.vic.gov.au/images/witness/rainier_verlaan/RV-1_4.jpg}}
    I gave a statement to police regarding the events of February 7, which is included in this statement as attachment RV-2.
    \footnote{\url{http://vol4.royalcommission.vic.gov.au/files/RV_2.pdf}}

    \subsection{Fire plan and preparations}

        Since purchasing our property, my wife and I have always focused on the bushfire risk in the area and have constantly prepared our property and house for the possibility of a bushfire. We moved to the area as we wanted to live on a bush block but we knew we had to modify the environment to accommodate the bushfire threat.
        When we first moved into the area, we read CFA literature and I attended two or three CFA meetings in our local area about how to prepare our property.
        My wife and I then came up with our bushfire survival plan. In the first two years of living in our home our plan was to leave, as we knew that our house and property were not prepared for a bushfire should one eventuate. Our three children were also younger then which was a consideration in choosing to leave. We knew that leaving meant leaving in the morning and having a designated place to go to, which might be a friend's place, until it had cooled down in the evening.
        Over the first few years of living in our house, we prepared our house and property so we would be able to defend it in the event of a bushfire.
        We installed a fire sprinkler system and fire fighting hoses at each end of our house. This was operated by a fire-fighting pump that was fed by a 50,000-litre concrete water tank that was kept full or nearly full throughout the bushfire season each year. The fire pump was petrol driven and housed in a brick enclosure with a corrugated iron roof to protect it during a bushfire. We tested the whole system at the beginning of each fire season.
        We cleared the undergrowth back for 30 - 40 metres around the house. This was done by hand as we didn't want to damage the natural native vegetation unnecessarily. We would do this every winter. We divided our property into four areas and would reduce the fuel of each quarter in a four-year rotation. We did this because we knew that reducing the undergrowth meant less risk of a fire crowning in the trees.
        We also developed a written fire plan, a copy of which is included in this statement as attachment RV-3.
        \footnote{\url{http://vol4.royalcommission.vic.gov.au/files/RV_3.pdf}}
         We kept this plan by the phone.
        Other preparations included:
        \begin{itemize}
            \item installing a 27,000 litre pool to supplement our water supply;
            \item ensuring that our insurance policy covered bushfire events;
            \item listening to the ABC using a transistor radio on total fire ban days;
            \item purchasing mobile phones; and
            \item purchasing a fireproof safe to store photos and documents and portable hard drives.
            \item keeping a completely cleared area of about 5 metres directly around the house.
        \end{itemize}
    \subsection{Black Saturday}

        I was aware a few days before February 7 that it was going to be a very hot day and on the evening of February 6, I was aware from the ABC news that it was going to be a total fire ban day. The Boolarra Delburn fires had been the week before which were about 20 to 25 kilometres from our house so I was well aware of the bushfires.
        During the week I ran the sprinklers through to check that everything worked and made sure that the gutters were cleaned out.
        On 7 February my wife and I were planning to stay home. We had the radio on all day to regional ABC.
        In the afternoon I heard on the radio that a fire had started near Churchill and was being fanned by strong northwesterly winds. I was also aware that there was a strong southwesterly change due later in the afternoon around 4.30 pm and that depending on the extent of the Churchill fire we would be under significant risk.
        I rang some teaching colleagues when I heard about the Churchill fire, and one of them, Karen Tingay said to me 'I can see the fire over the back ridge, I have to go now'. They live about 12 kilometres from us in a straight line in a WSW direction.
        At approximately 2.00 pm we decided to put our fire plan into action because we felt that something was going to happen. We blocked up our gutters, checked the fire pump and sprinkler system, moved items that were outside to our garage underneath the house and ensured we had a totally free area around the house. We stationed mops and buckets at strategic points around the house. We wet the house down and filled up the bath, sink, and water bottles. We also put towels in the bath, put wet towels on the windowsills, and changed into woollen fire fighting clothing.
        My daughter Chelsea had gone to the local river to cool down. She rang during the afternoon and I told her not to come home.
        At one stage my wife and I both asked each other whether we were sure that we wanted to stay. We both wanted to make sure that we were mentally capable of fighting the fire. We decided that we wanted to stay.
        By 5.00 pm, it was very dark and we were using a torch to walk down the veranda steps. It was clear by this time that a fire was approaching.
        About this time, my wife and I noticed that our neighbour's light was on and we went to check on her. Jan lives by herself and has no vehicle. She told us that another neighbour was driving her into Traralgon. Earlier, another neighbour Debbie had called us to say that she was leaving with her kids but that her husband Bill and their eldest boy were going to stay and defend their house.
        After that, we had both had something to eat because Ann said that we might be busy later.
        We kept listening to the radio all afternoon and into the evening but they did not mention Callignee as being under direct threat. They did mention that with the southwesterly change the fire would spread to the south of Callignee but they did not say anything about Callignee. However, we were ready for any fire that might eventuate. We spent a large amount of time waiting for the fire to come and thinking through what we had to do and double checking everything.
        It continued to get darker and then the power went off. There was a small ember attack about half an hour before the fire front came and then it stopped. We saw one neighbour leave at about this time. We were advised on the radio that the transmitter was going to go out and that we should switch to 828AM, which we did.
        The warnings of ABC radio were indicating that the fire was probably going to miss us to the south but the environmental conditions did not reflect this. We could see a bright orange haze, it was getting lighter, and then we heard a roaring noise. It was incredibly windy.
        At approximately 6.30 pm we came under ember attack. We immediately turned our fire-fighting pump on full with the sprinklers going on the roof and the back veranda and sprayers going over the timber windows and decking on the northeast and northwest faces of the house. We waited on the back veranda with mops and buckets to put out any embers that landed.
        Soon after the first embers landed, the fire front appeared over the ridge approximately 200 to 300 metres from our house. A large tree on our neighbour's property went up in flames. We noticed that our neighbour's house, the <This copy has been removed to protect the privacy of an individual.>, had caught on fire.
        As the front came closer, it reached the top of the trees, which were 20 to 30 metres tall, and the flames were well above the height of the trees. The front reached the fuel-reduced zone around our house and the fire continued to crown in the treetops although the fire's intensity was less at ground level.
        When the fire got to within 20 metres of our house, it was too hot to stay outdoors. We went inside to protect ourselves from the radiant heat. At this stage the embers were flying around our house horizontally and there was smoke everywhere.
        We checked that the sprinklers were still working every few minutes. We checked the roof cavities for any embers. The roof cavity above the bedroom was inundated with embers that were landing on the insulation bats. I used a wet towel put these out. My wife Ann was getting wet towels and checking other parts of the house under doors and windowsills for embers. By now, the house was full of smoke and the fire alarms were sounding continuously. We had wet scarves wrapped around our faces to help us breathe. While we were in the house, I looked outside and saw it was like a firestorm.
        After approximately 10-15 minutes, I noticed that the embers in the roof cavity had eased. I looked outside and saw that the sprinklers had stopped working. I thought that the worst of the fire front had passed us and so we ventured outside to do any mopping up that was required. I checked the sprinklers and discovered that the fire pump had burnt out because of ember attack and melting of some components.
        The house had not caught fire. The above ground pool was damaged extensively but still holding some water, and the pool decking and fence posts were on fire. The garden plants, shrubs and trees had all burnt or been singed. The garden mulch smouldered through the night. We also had an extensive woodpile 10 to 15 metres from the house that was on fire. Another stack of wood that was next to the fire-fighting tank was also on fire and had fallen towards the tank. By Sunday the tank did not have any water left in it as it was leaking out of the base.
        We were able to get some water from the house water tank to use for dousing areas that were still burning. Once the pool decking fire had receded, we were able to get access to the pool water to use for putting out spot fires.
        I patrolled the southwest side of the house, while Ann patrolled the northeast side. We used buckets of water and mops to put out spot fires.
        Our neighbour, Daniel <This copy has been removed to protect the privacy of an individual.>, and his brother came to check on us. He thought our house had burnt but it was the woodpile that he could see.
        Between approximately 9.00 pm and 9.30 pm I went to check on my neighbour Bill <This copy has been removed to protect the privacy of an individual.>. Bill and his eldest son had successfully defended their house. Bill also told us that he had seen other neighbour's, the <This copy has been removed to protect the privacy of an individual.> and <This copy has been removed to protect the privacy of an individual.>, leave just before the fire front came. We both went to check on our other neighbour Carol but her car was not in the carport and we assumed she had left.
        We continued to patrol the property all through the night and put out spot fires and flare ups with buckets and mops. At approximately 2.00 am we stopped for a break.
        At some stage during the night, a fire truck appeared on our road and stopped outside the <This copy has been removed to protect the privacy of an individual.> place but it did not come any further. I found this disconcerting as we were struggling to put out the woodpile that was concerning us. We could not put the woodpile out and it continued to burn the whole night and the next day. We just tried to protect the house against it.
        At approximately 4.00 or 5.00 am we gave each other a rest for an hour or so. I had ash in my eye that I couldn't remove but after a rest it got better. At sunrise we walked around the neighbourhood. We walked up to the neighbour's house that we had seen burning first. The house was completely burnt to the ground. We also saw that two houses across the road were completely burnt.
        I feel that we were lucky that the fire came through on the west side because the pool and the driveway created a barrier around the house. I also believe that the fuel reduction zone around the house helped protect the house because the intensity of the fire did lessen when it hit that zone.
        We did not call 000 at any point because we felt that there was not much point. We knew from our CFA training that we might not get any help and that we had to be independent.
        During the whole event, neither my wife nor I felt physically in danger. We knew that even if the house caught on fire, we could use the house to shelter from radiant heat.

    \subsection{Post Black Saturday}

        On Sunday 8 February, we made contact with our neighbours during the day. Our landline was still working, as were both our mobile phones. We rang friends and relatives to tell them what had happened, and that we were all right. We informed our work that we would not be in for a week.
        We rang our insurance company to make our claim that afternoon and managed to get through after an hour or so.
        Our daughter who had stayed in Traralgon on Saturday night was able to get back home through the roadblocks.
        There was no electricity. We had LP gas bottles for cooking but no running water or other amenities.
        On Monday 9 February there was still no power and our supplier informed us that it would take three to four weeks to have the power put back on. We managed to hire a generator from a local hire company. We had some diesel in the garage to keep it going for a while.
        On Monday, detectives from Victoria Police visited us about a fatality at the end of our street. We also had a visit from two people from the CFA who wanted to know what we had done to prepare and save the house.
        We knew that there were roadblocks in place at the top of Callignee South Road from the 'bush telegraph'. This made for a sense of isolation from the rest of the world in the first few days. We had access to radio but we had no TV as the fires had affected all the station's transmitters. It was not until Tuesday morning that we saw the TV and realised the devastation that had occurred in the rest of the state.
        We learnt on Tuesday, again via the 'bush telegraph', that local residents would be allowed back in to the area if they left through the roadblocks. We went into Traralgon to get some fresh supplies. We did not know until then that we had to register ourselves at the relief centre. We saw neighbours at the relief centre in Traralgon who had fled the fire and who we had not seen or heard from.
        We received amazing support from the Traralgon South community via the relief centre.

    \subsection{Changes in the future}

        In the future, we will still stay and defend if a fire comes again.
        We will improve some things including installing a sprinkler to protect the fire-fighting pump. We will look at how to protect our concrete tanks, and we will ensure that we block off the holes in the roof cavities. We are also looking at getting metal window shutters that roll down.
        We have had trees close to the house pushed over, because they could have fallen onto the house.

    \subsection{Stay or Go}

        Our position to stay and defend our property was based on our knowledge, ability and preparedness for such an event. If we have not been physically and emotionally prepared we would have left. By leaving, we know that means leaving early and by definition that means early in the mornings on all total fire ban days.
        In our street, all those who stayed to defend their property did so successfully. Those who left did not leave early and only left at the very last minute which was when they could see flames coming.
        I think that the education material on whether to stay or go should emphasise, in a much stronger way, the need to be mentally prepared.

    \subsection{Bushfire plans and building regulations}

        I suggest that people who live in bush areas should have to develop bushfire plans in consultation with the CFA or an authority, and should then have to register their bushfire plans to ensure that certain minimum requirements are met.
        All plans would have to be individual and designed to go with the environment that you are living in. The plans should take into account the surrounding bush, the elevation, the slope of the land, etc.
        I also think that it should be compulsory for people living in bushfire prone areas to attend CFA education meetings.
        I suggest that there should be one or two members of each CFA brigade or organisation whose sole responsibility would be to educate the community and be responsible for registration of bushfire plans. Every three years each property should have their fire plan reviewed with visits to the property to ensure that the plan can still be implemented properly.
        Building regulations and bushfire plans need to go hand-in-hand together. There is no point in having these bushfire building regulations without the need for some form of bushfire survival plan as well.
        There is no mention of fire sprinkler systems in the new regulations. These have shown to be very effective on many properties and I think that these should be a requirement to the higher levels of bushfire threat in the regulations.
        I believe there should be more leniency by the local councils in allowing people to clear vegetation that would improve the fire safeness of the house and the land around it. If a landowner is concerned about vegetation that might threaten their house they should have a right to clear that in consultation with the council.

    \subsection{Communications}

        I think that there has to be better communication given on the ABC and the commercial stations about the fire ban days and enacting personal bushfire plans. I believe there should be more explanation of enacting your bushfire plan.
        I also think that in the event of an emergency the relevant authorities should take over the ABC and commercial station bandwidth's and use all stations to broadcast emergency services information. This also needs to include TV.
        There needs to be better communication to victims after the event. As mentioned above, we were unaware that we had to register at the relief centre until Tuesday. I think there should be relief centre staff deployed to register people at their homes and make people aware of the services that are available. Alternatively, a mobile relief centre could be set up inside the roadblock areas for survivors to be able to attend and still return to their houses.
    \newline
    \newline
    Dated: June 2009
    \newline
    \newline
    RAINIER VERLAAN

\section{Anastasia Scott}

    I, Anastasia Scott, of <This copy has been removed to protect the privacy of an individual.>, Marysville, Victoria, state as follows:

    Ever since I left Greece as a little girl it had always been my wish to live in the mountains. I was attracted to Marysville by the fresh air, the serenity, the waterfalls, the beauty, the nature all around.
    In January 1995 I purchased land in <This copy has been removed to protect the privacy of an individual.>, Marysville and had a house built there – my dream home. The house was completed in March 1997. Over the years I put a lot of work into establishing an extensive garden. I lived there until 7 February 2009 when the house was destroyed in the Black Saturday fire.
    The house was known as "Anastasia Fairytale Cottage." From time to time I rented the house to tourists, to cover my mortgage payments and to pay for the gardener. A brochure advertising my cottage is Attachment 1 to this statement. The house was made of Western red cedar with a Colourbond roof.
    I sometimes opened my garden to raise money for charity. On one occasion I did this with other local gardens, and part of the collected funds went to SES and CFA.
    My block is directly opposite the Marysville DSE office in <This copy has been removed to protect the privacy of an individual.>. Before 7 February 2009 my neighbours were Alan Ryan at number 49, and the Fragga family, with a toddler and a newborn, at number 51. Our end of <This copy has been removed to protect the privacy of an individual.> is on the south western side of Marysville, on the side where the fire came first.

    \subsection{Fire preparation and planning}

    Before Saturday 7 February, I believed that the risk of fire in Marysville was possible, but remote. I had repeatedly been told that in 160 years Marysville had escaped bushfires. However, I never felt completely at ease that it would never happen.
    I was aware of the CFA's advice to stay and defend, or to leave early. Before 7 February 2009 I thought that the house could be defended, although at 63 years and 4 feet 11 inches I didn't think that I could defend it alone. I kept my gutters clear and the leaves swept from around the perimeter of the house, but my plan was always to go to the oval or the golf course if there was a fire.

    \subsection{Black Saturday}

    I knew that Saturday 7 February 2009 had been forecast to be a high fire danger. The weather forecasts were on the television, on ABC Radio 774 and in the newspapers. I was quite aware of the warnings that there would be high temperatures.
    I spent the day at home. Being a hot day, I closed the doors and the blinds and tried to keep cool. I was inside sewing for much of the day. Between 1.00 and 1.30 pm I noticed smoke. I was talking on the telephone to a friend, and said to her "I must go, I can see smoke on the horizon." I went outside and had a look. The smoke was not close - it could have been from the Kilmore fire. Smoke in this area is not unusual.
    My neighbour Alan Ryan, who lived at <This copy has been removed to protect the privacy of an individual.>, had told me he would have his radio on and that he would be in touch if there was anything to worry about.
    I continued sewing inside until about 5.00 pm when a friend who lives in Darwin Street, Marysville rang me up because he was aware that I live on my own. He said that a fire had started at Murrindindi Mill at 3 o'clock but it was 20 kilometres away. Initially I thought that he was telling me that it was too far away to be concerned. But then I thought if the wind is more than 100 km per hour it wouldn't take very long for the fire to travel 20 kilometres.
    I telephoned Alan next door. He said he had just come out of the shower and would be over. He didn't come immediately so I went over and told him what my friend had said. I kept on asking him "Is that flames in the sky?". By then the sky was becoming very cloudy, very smoky and you could see orange through it and I wasn't sure if it was the sun or flames. He told me he thought it was the sun, and I accepted what he said as he is more knowledgeable about weather than I am.
    Then we went back to my house and he said he would clean my gutters. His house was a tiny little house, it was under so many trees, it would have taken him a long time to clean out his gutters, and there was nothing that he could do there to save the house if the fire came. So he came over to help me because we had more of a chance to save my house, and if we couldn't save my house, then we couldn't save next door.
    Alan then cleaned out my gutters. Because of the gutter guard, all he had to do was to take a long broom and sweep the leaves off. The gutters had been cleaned that morning, but dry leaves in the strong wind had collected on the gutter guard only hours after. I started hosing the house down and sweeping up the fallen leaves. We were out here, hosing down the walls and filling up large containers with water. We found margarine and ice cream containers so that we could empty the water to put out embers.
    I had put dinner on the stove to cook it early, and I went inside to turn it off. I found that the stove was off and there was no electricity. This was at about 5.30 to 5.45 pm. A tree had blown down nearby and had brought down the wires, thus no electricity.
    By then my neighbour and I were aware that danger was approaching, and that we could not defend the property. However, with electricity off we couldn't open the garage door by remote control to get the car out of the garage. I found the keys to the garage, but could not open the door with the keys either.
    I have always heard the CFA siren when they sounded it for practices. That day I didn't hear it. Supposedly they did sound it, for only 2 brief bips. This siren should have been sounding continuously to alert us.
    Just after 6 o'clock, we were on the driveway. A lady who works at DSE, Gillian Pennant, came over and said "You must go down to the Cumberland." I said "Is it that serious?" She said, "Yes, yes, everybody is already there." I wondered why everybody was there and I didn't know about it. Alan said to me "Go to the Cumberland and I will follow you". So I walked down <This copy has been removed to protect the privacy of an individual.> towards the Cumberland.
    On my way I passed the tree that had fallen on top of a car, and there were SES and Ambulance vehicles trying to free the driver in the car. Someone there told me that half of Narbethong was burnt. I rang the friend who had called me at 5.00 pm to tell him that half of Narbethong was burnt and to please go. That phone call is registered on my Telstra account at 6.11 pm.
    When I reached the Cumberland, it was deserted. It was very odd and eerie. There were only one or two cars parked in front of it and I thought how can that be, everybody is supposed to be here. The oval is very close to the Cumberland. Because there were no cars at the Cumberland I assumed that there was no-one at the oval either. Then Alan came down and said he would go back to the house and see if he could open the garage door with a crowbar.
    Alan took me to the retirement village where Mary and Reg Kenealy were, and then he went back to my house. As I waited for Alan to return I became concerned that Alan might be trapped in the garage, and I tried to ring him on his mobile phone from the landline at the retirement village. He did not answer and the call is recorded as a missed call at 6.37 pm. Mr and Mrs Kenealy's son had taken their car, his had run out of petrol, and they tried to ring him to tell him to bring the car back, but the mobile phones did not operate.
    By then the roar of the approaching fire was deafening and I felt extremely apprehensive. Alan could have been trapped in the way of the fire, possibly without a car, and I was with a couple who were not prepared to leave. Mr Kenealy is the President of the Historical Society, and he wanted to go and hose down the museum. Mrs Kenealy had her trailer packed with all the historical mementoes and treasures, but she couldn't pull it with the car, it was too heavy, so she wanted to stay and hose it down. And then something came down and smashed against the window and she said "Here is the first one". They were actually live embers and she grabbed the hose across the driveway and started hosing them.
    Then her son arrived and asked "What's that roar?" and she said "That's the firestorm, Marysville is going to go." Some minutes later, Alan arrived in the driveway of the retirement village in my car. I went in the car with him and we drove in very dense smoke to the Buxton-Marysville Road. As we left inside me I was willing Mr and Mrs Kenealy to forget the Historical Society and get in the car with their son and go. They did and drove out shortly after us.
    I did not see flames, there was too much smoke. I could see sparks, not as we started to move out of town but there were sparks when we were in the town itself. In the distance from the retirement village to the start of the Buxton Road, you could see sparks and branches and hear the general uproar of the fire. I assume we drove faster than the fire, and we escaped it.
    We could not see any vehicles either in front of us or behind us when we drove out of Marysville. Initially it was too smoky, too dense to see anything until we had travelled about a kilometre along Buxton Road and then we couldn't see anyone in front or behind. When we arrived at the golf course, which we thought was the next point of evacuation, an officer from the police or from the CFA, was frantically waving for us to go, not to stop, to keep on going.
    We drove out of Marysville very close to 7 o'clock. The convoy that left from the oval was nowhere to be seen by us. I believe the convoy had already left. We could not see any cars in front of us or behind us on our drive from Marysville to Buxton and then we drove to Maroondah Highway to Alexandra. We only saw one vehicle going into Marysville not far away from the caravan park.
    As we were driving along the Buxton Road I tried to ring 000. I could not get through. That may be because the mobile phones were not working by then. I do not remember sending it but an SMS to 000 is still registered on my mobile as a failed message.
    Then we drove to Alexandra, went to the shire offices, and asked where we could go to spend the night and they sent us to the local school.
    I knew in my heart that my cottage had gone. When I arrived at the Red Cross at the school in Alexandra, we were among the last to arrive. A lot of people were there. I saw the lady who cleaned the cottage, she was registering with the Red Cross. She looked at me and she had a questioning look in her eyes and without her asking me, I said "It's gone", meaning my cottage was gone. And she turned her head around and cried. I had resigned myself to the fact that my dream house had gone.
    On the first day that we were able to return back to Marysville in the buses Peter Collyer, the local policeman, told me that he had evacuated the family of another local policeman, Ian Thompson. The family lived on the other side of <This copy has been removed to protect the privacy of an individual.>, in the house on the corner one door down from the DSE office, with his wife Kristy and two small boys. Police officer Ian Thompson warned earlier an elderly gentleman Jim Sherlock twice. These warning opportunities by police on duty were random, they warned some and not others. There were six people on <This copy has been removed to protect the privacy of an individual.>, myself, Alan and the Fragga family with a 3 week old baby. There were two people who lived opposite Mr Sherlock on <This copy has been removed to protect the privacy of an individual.>. Three people lived below me, an elderly couple and a lady on her own. The only warning they had was the two short CFA siren bips. They evacuated when their homes started to shake violently. Their only exit was their common driveway, a fallen tree would have trapped them (they are so traumatised they will not return to Marysville). It has worried me, and I question the reason why policemen, on duty, did not warn us. They had police cars, they could have driven along our streets with sirens on and lights flashing and turned easily back. Both <This copy has been removed to protect the privacy of an individual.> and <This copy has been removed to protect the privacy of an individual.> have areas where vehicles can make U turns. The combination of smoke, wind, and such warnings on police vehicles would have alerted us sooner. If one of us was warned earlier we would have warned the others.
    Alan told me that the Fragga family left a few minutes before he drove down to the retirement village. They are all safe, but their house burned down.

    After Black Saturday

    My cottage was insured. Apart from the actual loss of life – four neighbours on this street – my personal loss was the garden that was not insured. The garden cost substantially more than the cottage and it will take 10 to 12 years to establish to its condition prior to the fire. This is not possible for me to redo.
    Most of my income came from renting out my cottage to tourists and another little house I had in Marysville that also burned. I do not qualify for the pension. Until I can rebuild the only income is from a small superannuation, which is not enough to meet my basic needs
    Rebuilding is very difficult. I am in contention with my insurance company about the amount they will pay for the destroyed houses. I have had an assessment done of my bushfire attack level, it is flame zone. When the site has been cleared of dangerous trees the assessment may change. The tree loppers won't go to work until sites next door are cleared of asbestos. So I don't know where I stand and it is all taking so long. I have seen a builder who has done the preliminary floor plans but I can't really proceed much beyond that. He can't give me a quote they are only valid for 30 days and then the prices rise. I can't sign a contract. The longer it takes, the harder it seems and the longer I am without home and sufficient income. The houses cannot be reassessed for bushfire attack levels until the neighbours cut their trees which is out of my control. If rebuilding does not commence or another property purchased within 12 months of the Insurance payout there is Capital Gains Tax and Stamp Duty to be paid for purchases elsewhere, but without the value of land in cash this is not possible. We did not choose to be in this situation and fail to understand why the ATO and SRO cannot waive these high expenses as a result only of the fire, which are obstacles to rebuild and resume our lives. Can the authority who imposes the extra costs upon us with the BAL building regulations assure us that our homes will not be destroyed in future similar fires?
    My neighbour Alan was renting the house next door. He does not know whether the owners will rebuild and whether he will be able to rent there when they do. At present he is living in a rented house in Alexandra. I share this house with him until I am given accommodation at the temporary village.
    That's how it has impacted on me, who was lucky enough not to lose family members. I can't walk down the street without crying when I pass the houses of the people we lost.
    I believe the largest contributor to this tragedy has been the lack of preventative measures to backburn and reduce fire fuel in the forests. No large quantities of fire fuel in the forests means no large bushfires.
    \newline
    \newline
    ANASTASIA SCOTT

\end{document}
